\documentclass{article}
\nofiles
\usepackage{minted}
\usepackage{dirtree}
\usemintedstyle{colorful}

\title{Toscop}
\author{Fabio Kleis \textlangle fabiohrkc@gmail.com\textrangle}
\date{\today}
\begin{document}
\maketitle

Toscop é um dashboard de processos para linux escrito em linguagem C. 
O projeto utiliza bibliotecas padrão da linguagem C, como \textit{pthread}
para programação em multithreading e \textit{ncurses} para programacão
de uma \textit{tui}. Segue uma organização simples separando um diretório
para arquivos .h e outro diretório para arquivos .c, utiliza o sistema make
para o gerenciamento de compilação. A estrutura do projeto:

\dirtree{%
.1 toscop/.
.2 Makefile\DTcomment{GNU Makefile}.
.2 LICENSE\DTcomment{MIT LINCESE}.
.2 README.md.
.2 include/\DTcomment{C include directory}.
.3 proc\_list.h.
.3 proc\_parser.h.
.3 term\_header.h.
.3 term\_procs.h.
.3 toscop.h.
.3 toscop\_thread.h.
.3 toscop\_win.h.
.3 w\_proc.h.
.2 src/\DTcomment{C source directory}.
.3 main.c.
.3 proc\_list.c.
.3 proc\_parser.c.
.3 term\_header.c.
.3 term\_procs.c.
.3 toscop.c.
.3 toscop\_thread.c.
.3 toscop\_win.c.
.3 w\_proc.c.
}
\hfill \break
Antes de prosseguirmos com as explicações do código alguns exclarecimentos 
sobre a organizaçãdo do documento devem estar ditas. Serão suprimidos comentários
demasiadamente longos assim como espaços desnecessários e algumas explicações gerais
sobre a linguagem C que podem estar presentes no código fonte do projeto mas não
estaram presente neste documento.


O começo do programa está no arquivo \textit{toscop/src/main.c}, nele são chamadas as
funções para tratar os argumentos via linha de comando, inicializar o programa
e por fim saír.

\begin{minted}{c}
#include "toscop.h"

int main(int argc, char** argv) {
    cli(argc, argv);
    run();
    pthread_exit(NULL);
}
\end{minted}

\section{threads}
O programa utiliza duas threads para separar suas funcionalidades, sendo 
mostrar as informações na tela e atualizar suas informações, as threads
são gerenciadas via variáveis globais no arquivo \textit{toscop/include/toscop.h}:

\begin{minted}{c}
#define MUTEX_FUNC(MUTEX, FUNC, ...) do \
    { pthread_mutex_lock(MUTEX); FUNC(__VA_ARGS__); pthread_mutex_unlock(MUTEX); } \
while(0)

extern pthread_mutex_t toscop_mutex;   // mutex global
extern double max_time;                // tempo maximo para cada refresh
extern bool fdebug;                    // flag de debug
extern term_header* th;                // gerenciador das informacoes globais
extern term_procs* tp;                 // gerenciador da lista de procs
extern toscop_wm* wm;                  // gerenciador de window para cada info
extern void run(void);
extern void cli(int argc, char** argv); 
\end{minted}

O programa divide as duas principais funcionalidades nas threads
print\_th e refresh\_th, a estrutura das threads e suas funcões estão 
declaradas no arquivo \textit{toscop/include/toscop\_thread.h}:

\begin{minted}{c}
typedef struct toscop_thread_t {
    pthread_t thread_id;
} toscop_thread_t;

extern void* print_th(void* arg);
extern void* refresh_th(void* arg);
\end{minted}

São definidas variávies static no começo do arquivo \textit{toscop/src/toscop\_thread.c}
 para o controle da execução das threads e o tempo de atualização das informações globais.

\begin{minted}{c}
static int k_p = 0;
static double refresh_t = 0;
\end{minted}

\subsection{print\_th}

A thread print\_th é responsável por mostrar as informações contidas
em cada uma das estruturas declaradas globalmente no arquivo 
\textit{toscop/include/toscop.h}, a funcão print\_th está definida 
no arquivo \textit{toscop/src/toscop\_thread.c}:

\begin{minted}{c}
void* print_th(void* arg) {
    (void) arg; 

    clock_t pr_t = clock();
    double p_t = 0;

    while (k_p != 'q') {

        // muda o estado interno do do wm com base no k_p
        MUTEX_FUNC(&toscop_mutex, handle_key, wm, &k_p);

        p_t = (double)(clock() - pr_t) / CLOCKS_PER_SEC;

        if (p_t >= 0.2) { // a cada 0.2 segundos é atualizado o print
       
            MUTEX_FUNC(&toscop_mutex, show_toscop, wm);
         
            // mostra informacoes de debug (configurado via flag -v)
            if (fdebug)
                MUTEX_FUNC(&toscop_mutex, show_debug_info, wm, refresh_t);

            pr_t = clock();  // reseta clock
        } 
    }
    pthread_exit(NULL);
}
\end{minted}

A condição de parada é definida quando a tecla `q' é pressionada, 
podendo assim ser feito o pthread\_join. A funcionalidade dessa thread 
é mostrar todas as informações do programa através da função show\_topscop 
que é envelopada na macro MUTEX\_FUNC que força essa chamada de função
funcionar com exclusão mútua através do mutex global toscop\_mutex.

\pagebreak

\subsection{refresh\_th}

A thread refresh\_th é responsável por atualizar as informações contidas em cada
uma das estruturas declaradas globalmente no arquivo \textit{toscop/include/toscop.h}, 
a funcão refresh\_th está definida no arquivo \textit{toscop/src/toscop\_thread.c}:

\begin{minted}{c}
void* refresh_th(void* arg) {
    (void) arg; 
    
    struct timespec st = {0}, ct = {0};
    clock_gettime(CLOCK_MONOTONIC, &st); // pega o tempo deis do boot para o start_time
    cpu_stats last_stat = {0}; 
 
    while (k_p != 'q') {

        clock_gettime(CLOCK_MONOTONIC, &ct); 
        // calcula o tempo atual - tempo do começo
        refresh_t = (ct.tv_sec - st.tv_sec) + ((ct.tv_nsec - st.tv_nsec) / N_TO_S); 

        // caso tenha passado max_time sec da refresh 
        if (refresh_t >= max_time) {
            st = ct;                  // reseta o clock, start time recebe current time
            last_stat = th->cpu_stat; // guarda o ultimo valor de cpu usage sem delta
      
            pthread_mutex_lock(&toscop_mutex);

            tp_free(tp);                    // limpa lista de procs anterior
            tp = create_term_procs();       // cria nova lista
          
            th_free(th);                    // limpa informacoes globais anterior
            th = create_term_header();      // cria o novo term header
            calc_cpu_stats(th, last_stat);  // calcula delta de % cpu
           
            pthread_mutex_unlock(&toscop_mutex);
        
        }
    } 
    pthread_exit(NULL);
}
\end{minted}

A condição de parada é a mesma da thread print\_th. A thread refresh\_th atualiza
as informações das estruturas globais de forma periódica, sendo controlado os ciclos 
pela variável global max\_time.

\pagebreak

\end{document}
