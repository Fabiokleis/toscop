\documentclass{article}
\nofiles
\usepackage{minted}
\usepackage{dirtree}
\usepackage{graphicx}
\usemintedstyle{colorful}

\title{Toscop}
\author{Fabio Kleis fabiohrkc@gmail.com}
\date{\today}
\begin{document}
\maketitle

Toscop é um dashboard de processos para linux escrito em linguagem C. 
O projeto utiliza syscalls do linux e bibliotecas padrão da linguagem C
como \textit{pthread} para programação em multithreading e \textit{ncurses}
para programacão de uma \textit{tui}, segue uma organização simples 
separando um diretório para arquivos .h e outro diretório para arquivos .c,
utiliza o sistema make para o gerenciamento de compilação:  

\dirtree{%
.1 toscop/.
.2 include/\DTcomment{C include directory}.
.3 proc\_list.h.
.3 proc\_parser.h.
.3 term\_header.h.
.3 term\_procs.h.
.3 toscop.h.
.3 toscop\_thread.h.
.3 toscop\_win.h.
.3 w\_proc.h.
.2 LICENSE\DTcomment{MIT LICENSE}.
.2 Makefile\DTcomment{GNU Makefile}.
.2 README.md.\DTcomment{README PLS}.
.2 src/\DTcomment{C source directory}.
.3 main.c.
.3 proc\_list.c.
.3 proc\_parser.c.
.3 term\_header.c.
.3 term\_procs.c.
.3 toscop.c.
.3 toscop\_thread.c.
.3 toscop\_win.c.
.3 w\_proc.c.
}
\hfill \break
Antes de prosseguirmos com as explicações do código alguns exclarecimentos 
sobre a organizaçãdo do documento devem estar definidas. O objetivo é mostrar
de forma resumida as estruturas principais do projeto e seu funcionamento
como um todo. Então serão suprimidas muitas informações que podem estar
presentes no código fonte do projeto mas não estaram presente neste documento.

O começo do programa está no arquivo \textit{toscop/src/main.c}, nele é chamada
função cli para tratar os argumentos via linha de comando e parametrizar as variáveis
globais, e também a função run que inicializa as estruturas globais, cria as threads e
faz o join delas, e por fim termina o processo.

\begin{minted}{c}
#include "toscop.h"

int main(int argc, char** argv) {
    cli(argc, argv);
    run();
    pthread_exit(NULL);
}
\end{minted}

\section{Threads do pthread}
O programa utiliza duas threads para separar suas funcionalidades, sendo 
mostrar as informações na tela e atualizar suas informações, as threads
são gerenciadas via variáveis globais no arquivo \textit{toscop/include/toscop.h}:

\begin{minted}{c}
#define MUTEX_FUNC(MUTEX, FUNC, ...) do \
    { pthread_mutex_lock(MUTEX); FUNC(__VA_ARGS__); pthread_mutex_unlock(MUTEX); } \
while(0)

extern pthread_mutex_t toscop_mutex;   // mutex global
extern double max_time;                // tempo maximo para cada refresh
extern bool fdebug;                    // flag de debug
extern term_header* th;                // gerenciador das informacoes globais
extern term_procs* tp;                 // gerenciador da lista de procs
extern toscop_wm* wm;                  // gerenciador de window para cada info
extern void run(void);
extern void cli(int argc, char** argv); 
\end{minted}

O programa divide as duas principais funcionalidades nas threads
print\_th e refresh\_th, a estrutura das threads e suas funcões estão 
declaradas no arquivo \textit{toscop/include/toscop\_thread.h}:

\begin{minted}{c}
typedef struct toscop_thread_t {
    pthread_t thread_id;
} toscop_thread_t;

extern void* print_th(void* arg);
extern void* refresh_th(void* arg);
\end{minted}

São definidas variávies static no começo do arquivo \textit{toscop/src/toscop\_thread.c}
 para o controle da execução das threads e o tempo de atualização das informações globais.

\begin{minted}{c}
static int k_p = 0;
static double refresh_t = 0;
\end{minted}

\subsection{Thread print\_th}

A thread print\_th é responsável por mostrar as informações contidas
em cada uma das estruturas declaradas globalmente no arquivo 
\textit{toscop/include/toscop.h}, a funcão print\_th está definida 
no arquivo \textit{toscop/src/toscop\_thread.c}:

\begin{minted}{c}
void* print_th(void* arg) {
    (void) arg; 

    clock_t pr_t = clock();
    double p_t = 0;

    while (k_p != 'q') {

        // muda o estado interno do do wm com base no k_p
        MUTEX_FUNC(&toscop_mutex, handle_key, wm, &k_p);

        p_t = (double)(clock() - pr_t) / CLOCKS_PER_SEC;

        if (p_t >= 0.2) { // a cada 0.2 segundos é atualizado o print
       
            MUTEX_FUNC(&toscop_mutex, show_toscop, wm);
         
            // mostra informacoes de debug (configurado via flag -v)
            if (fdebug)
                MUTEX_FUNC(&toscop_mutex, show_debug_info, wm, refresh_t);

            pr_t = clock();  // reseta clock
        } 
    }
    pthread_exit(NULL);
}
\end{minted}

A condição de parada é definida quando a tecla `q' é pressionada, 
podendo assim ser feito o pthread\_join. A funcionalidade dessa thread 
é mostrar todas as informações do programa através da função show\_topscop 
que é envelopada na macro MUTEX\_FUNC que força essa chamada de função
executar com exclusão mútua através do mutex global toscop\_mutex.

\pagebreak

\subsection{Thread refresh\_th}

A thread refresh\_th é responsável por atualizar as informações contidas em cada
uma das estruturas declaradas globalmente no arquivo \textit{toscop/include/toscop.h}, 
a funcão refresh\_th está definida no arquivo \textit{toscop/src/toscop\_thread.c}:

\begin{minted}{c}
void* refresh_th(void* arg) {
    (void) arg; 
    
    struct timespec st = {0}, ct = {0};
    clock_gettime(CLOCK_MONOTONIC, &st); // pega o tempo deis do boot 
    cpu_stats last_stat = {0}; 
 
    while (k_p != 'q') {

        clock_gettime(CLOCK_MONOTONIC, &ct); 
        // calcula o tempo atual - tempo do começo
        refresh_t = (ct.tv_sec - st.tv_sec) + ((ct.tv_nsec - st.tv_nsec) / N_TO_S); 

        // caso tenha passado max_time sec da refresh 
        if (refresh_t >= max_time) {
            // reseta o clock, start time recebe current time
            st = ct;                  
            // guarda o ultimo valor de cpu usage sem delta
            last_stat = th->cpu_stat; 
      
            pthread_mutex_lock(&toscop_mutex);

            tp_free(tp);                    // limpa lista de procs anterior
            tp = create_term_procs();       // cria nova lista
          
            th_free(th);                    // limpa informacoes globais anterior
            th = create_term_header();      // cria o novo term header
            calc_cpu_stats(th, last_stat);  // calcula delta de % cpu
           
            pthread_mutex_unlock(&toscop_mutex);
        
        }
    } 
    pthread_exit(NULL);
}
\end{minted}

A condição de parada é a mesma da thread print\_th. A thread refresh\_th atualiza
as informações das estruturas globais de forma periódica com exclusão mútua
utilizando as funções pthread\_mutex\_lock e pthread\_mutex\_unlock,
sendo controlado os ciclos pela variável global max\_time.

\section{Telas do ncurses}
O programa define um gerenciador de janelas toscop\_wm que 
separa as informações coletadas por telas, em ordem, começando
com informações globais, em seguida uma lista com todos os processos lidos e
por fim uma tela com informações detalhadas de um processo:
\begin{enumerate}
    \item t\_win th\_win 
    \item t\_win tp\_win  
    \item t\_win proc\_win 
\end{enumerate}

Seguindo a mesma enumeração, as estruturas utilizadas para guardar os dados são:
\begin{enumerate}
    \item term\_header th 
    \item term\_procs tp
    \item w\_proc proc
\end{enumerate}

A estrutura de tela utilizada está declarada no arquivo 
\textit{toscop/include/toscop\_win.h}:
\begin{minted}{c}
// wrapper da window com suas informacoes basicas
typedef struct t_win {
    WINDOW* win;
    int width;
    int height;
    int x;
    int y;
    uint64_t starts_at;
} t_win;
// enum para alternar entre telas
typedef enum WIN_TABS {
    TH_WIN = 0,
    TP_WIN,
    PROC_WIN,
} WIN_TABS;
// estrutura para genrenciamento das telas do ncurses
typedef struct toscop_wm {
    t_win th_win;           // term_header window (info global)
    t_win tp_win;           // term_procs window  (info da lista de procs)
    t_win proc_win;         // window do proc     (info do processo)

    WIN_TABS c_win;         // window atual que esta com foco
} toscop_wm;
\end{minted}

\subsection{Tela th\_win}
A primeira tela do toscop mostra informações globais do sistema 
operacional. Todas as informações foram obtidas via \textit{syscalls}
do linux ou lendo o diretório \textit{/proc} e seus subdiretórios. A 
estrutura que guarda todas as informações da tela th\_win, é term\_header
que está declarada em \textit{toscop/include/term\_header.h} 

\begin{minted}{c}
// cabeçalho do toscop com informações globais
typedef struct term_header {
    struct sysinfo si;  // sysinfo contem informaçoes do sistema
    struct tm* ti;      // timeinfo contem informacoes de tempo

    cpu_stats cpu_stat; // valores de tempo de uso do cpu, em secs e em %
    mem_stats mem_stat; // valores de memoria, em uso, livre, virtual etc...

    double lavg[3];     // loadavg do sistema

    uint64_t d_uptime;  // tempo em sec dos dias
    uint64_t h_uptime;  // tempo em sec das horas
    uint64_t m_uptime;  // tempo em sec dos minutos

    uint64_t t_procs;   // total de processos listados no /proc/[pid]
    uint64_t t_threads; // total de tasks listados no /proc/[pid]/task/[tid]
} term_header;
\end{minted}
nota: si é um struct utilizado na \textit{syscall sysinfo};\\
nota: ti é o struct retornado da funcão localtime utilizando a \textit{syscall time};\\
nota: cpu\_stats é um struct que guarda informações do \textit{/proc/stat};\\
nota: mem\_stats é um struct que guarda informações do \textit{/proc/meminfo};

\begin{figure}[hb]
    \centering
    \includegraphics[width=1\textwidth]{toscop\_header}
    \caption{exemplo tela th\_win}
\end{figure}
As informações do sistema operacional estão organizadas em linhas, 
contendo data e hora, número de processos e threads, estados dos processos, 
porcentagem de uso de CPU e de memória, memória RAM e memória virtual,
memória swap e memória compartilhada, tamanho de página e memória em buffer/cache.

\subsection{Tela tp\_proc}
A segunda tela do toscop mostra uma lista de processos. Todas as 
informações para criar a lista são lidas no \textit{/proc/[pid]} e seus
subdiretórios, a estrutura que guarda todos processos é term\_procs declarada
em \textit{toscop/include/term\_procs.h} que contém a lista duplamente
encadeada declarada em \textit{toscop/include/proc\_list.h}:
\begin{minted}{c}
// lista duplamente encadeada para guardar a struct do proc
typedef struct proc_list {
   w_proc *proc;
   struct proc_list *next;
   struct proc_list *prev;
} ProcList;
// estrutura que le o /proc e inicializa a lista de procs
typedef struct term_procs {
    ProcList *proc_list;
    ProcList *proc_list_tail;
} term_procs;
\end{minted}
\begin{figure}[hb]
    \centering
    \includegraphics[width=1\textwidth]{toscop\_procs}
    \caption{exemplo tela tp\_win}
\end{figure}
As informaçoes dos processos estão separadas por colunas sendo:
identificador do processo; nome do usuário dono do processo;
prioridade real no escalonador; prioridade em espaço de usuário;
estado do processo; arquivo executável que iniciou o processo.

\subsection{Tela proc\_win}
A terceira tela do toscop mostra informações detalhadas de um processo. Todas
as informações para criar um processo são lidos no \textit{/proc/[pid]} e seus
subdiretórios, a estrutura que guarda um processo é w\_proc declarada
em \textit{toscop/include/w\_proc.h}:
\begin{minted}{c}
// conteudo de um processo
typedef struct w_proc {
    token* ptokens;          // campos do /proc/[pid]/stat
    char* path;              // caminho absoluto -> /proc/[pid]
    char* owner_name;        // nome do usuario dono da proc
    int uid;                 // user id
    int gid;                 // grupo dono da proc

    uint64_t r_mem;          // (rss) ram total de memoria do processo 
    uint64_t v_mem;          // virtual size total do processo
    uint64_t pr_mem;         // total de paginas de rss do processo
    uint64_t pv_mem;         // total de paginas de vm do processo
    uint64_t stack_pages;    // total de paginas da stack
    uint64_t stack_size;     // total em kB da stack
    uint64_t heap_pages;     // total de paginas da heap
    uint64_t heap_size;      // total em kB da heap
    uint64_t text_pages;     // total de paginas do text
    uint64_t text_size;      // total em kB do text

    uint64_t r_threads;      // running threads de um proc
    uint64_t s_threads;      // sleeping threads de um proc 
    uint64_t z_threads;      // zoombie threads de um proc
    uint64_t i_threads;      // idle threads de um proc
} w_proc;
\end{minted}
nota: token *ptokens é um struct declarado em \textit{toscop/include/proc\_parser.h} que guarda os campos definidos no \textit{/proc/[pid]/stat};
\begin{figure}[hb]
    \centering
    \includegraphics[width=0.5\textwidth]{toscop\_proc}
    \caption{exemplo tela proc\_win}
\end{figure}

A tela do processo está organizada em linhas, contendo identificador e
nome do comando que inicializou o processo, número total
de threads e seus estados, memória virtual, memória stack, memória
heap e memória de text/code, tanto em kB quanto em número de páginas.

\end{document}
